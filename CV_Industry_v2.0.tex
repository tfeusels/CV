\documentclass[11pt,a4paper,oneside,fleqn]{article}

\usepackage{a4wide}
\usepackage[dutch,english]{babel}			%Nederlandse hyphenatie
\usepackage{amsmath}					%Meer wiskundemogelijkheden
\usepackage{amssymb}					%Meer symbolen en speciale letters
\usepackage[latin1]{inputenc}				%om niet ascii karakters in de code te kunnen typen
\usepackage{fancyhdr}					%om de paginanummering en headers naar mijn hand te zetten
\usepackage[small,bf,hang]{caption}
\usepackage{float}
\usepackage{xspace}
\usepackage[Gray,squaren,thinqspace,thinspace]{SIunits}
\usepackage{fontenc}
\usepackage{hyperref}

\newcommand{\ignore}[2]{\hspace{0in}#2}
\pagestyle{fancy}
\fancyhf{}
\renewcommand{\headrulewidth}{0pt}
\renewcommand{\footrulewidth}{0pt}
\fancyhf[HL]{\nouppercase{\textit{\leftmark}}}
\fancyhf[FC]{\thepage}
\headheight 14pt
\hyphenation{TRIUMF}
% arXiv$\:$-1805.-04163-[phys-ics.-ins-det]}

\begin{document}
\title{Curriculum Vitae }
\author{Tom Feusels}
\date{}  
\maketitle
%\section{Personal Details}
Tom Feusels\\
Jef van Hoofplein 11\\
2530 Boechout\\
Belgische nationaliteit\\
+32 483 65 99 09/0483 65 99 09\\
feusels.tom@gmail.com
%\begin{description}
%\item[Name:] Feusels
%\item[First Name:] Tom
%\item[Born:] August 20, 1985, Sint-Niklaas, Belgium
%\item[Citizenship:] Belgian
%\item[Address:] Jef Van Hoofplein 11, 2530 Boechout, Belgie
%%3842 West 16th Avenue, Vancouver, BC V6R 3C7, Canada 
%%De Gravestraat 55, 9100 Sint-Niklaas, Belgie
%%1508-1255 Bidwell St, Vancouver BC V6G 2K8, Canada
%%Jef Van Hoofplein 11, 2530 Boechout, Belgie
%\item[E-mail:] feusels.tom@gmail.com %tom.feusels@ugent.be %feusels.tom@gmail.com
%\item[Phone:] +32 483 65 99 09. 
%+1 604 441 3964
%0032/486 28 94 29.
%\end{description}
\section{Education}
Academic studies:\\
%\begin{itemize}
%\item 2003 - 2007: Bachelor and Master studies in Physics and Astronomy at the University of Gent, Belgium.
%\item 2007 - 2013: Doctoral Studies at the University of Gent, Belgium. 
%\item 2013 - 2018: Post-Doctoral Fellowship at the University of British Columbia, Vancouver, Canada.
%\end{itemize}
2003 - 2007: Bachelor and Master studies in Physics and Astronomy at the University of Gent, Belgium.\\
2007 - 2013: Doctoral Studies at the University of Gent, Belgium.\\
2013 - 2018: Post-Doctoral Fellowship at the University of British Columbia, Vancouver, Canada.

%Academic degrees:
%\begin{itemize}
%\item First Year First Cycle Mathematics and Physics : degree on July 7, 2004 : with Distinction
%\item Second Year First Cycle Physics : degree on July 6, 2005 : with Distinction
%\item First Year Second Cycle Physics (2005-2006) : degree on July 5, 2006 : with Distinction
%\item Second Year Second Cycle Pysics (2006-2007) : degree on July 4, 2007 : with Greatest Distinction 
%      (eigenlijk niet van toepassing omdat er geen graden meer zijn per jaar, hier 847/1000)
%\item Bachelor of Physics and Astronomy on July 6, 2005: with Distinction
%\item Master of Physics and Astronomy on July 4, 2007: with Great Distinction
%\end{itemize}

%Belangrijke punten die er uit moeten springen:
%- PhD + Post-Doc
%-- publicaties
%-- teaching/supervisor experience
%- PC kennis
%-- programmeertalen - je hebt een hele lijst, haal er degene uit waar je echt pro efficient in bent
%-- Grid experience




\section{Skills}
\begin{itemize}
\item Languages: 
\begin{itemize}
\item Dutch: native
\item French: moderate %good
\item English: excellent
\item German: basic
%\item Japanese: spoken/written: basic
\end{itemize}
\item Computer Knowledge: 
\begin{itemize}
\item Operating Systems: Windows, Linux, Mac OS X.
\item Text editing software: LaTeX, Emacs, Vi(m).
%MS/Open/Libre Office, Google Docs(Docs, Sheets, Slides), LaTeX, Emacs, Vi(m), doxygen.
\item Version control software: CVS, SVN, Git.
\item Programming languages: Java, JavaScript, C, C++, Python, Fortran, %(77, 90, 95 and 2003)
   PHP.
\item Python: Pandas, Numpy, Scipy, Seaborn, SQLite, SQLAlchemy, Basemap, Matplotlib, Jupyter Notebook, Scikit-learn,
      IPython, Spyder.
(Tensorflow, PyTorch, Keras).
\item ((Apache: Hadoop, Spark, Kafka, Flink, Storm, Samza, Cassandra, Elasticsearch, Hive))
\item ((Experience with Amazon Web Services, Spark, Splunk, Kubernetes, Docker, Helm, iPass, Grafana, Prometheus, NGNIX.))


%\item (CPU parallel development: MPI, OpenMP, Chapel)
%\item (GPU parallel development: CUDA, OpenCL)
%\item Scientific software: LabVIEW, ROOT, TMVA, GSL, Geant4, NEUT, Genie, NuWro, GiBUU.
\item Job Scheduling: (Open)PBS cluster, TORQUE, HTCondor, Ganglia, Maui Cluster Scheduler, Moab Cluster,
Valgrind. 
%Valgrind(Massif, Cachegrind $\rightarrow$ KCacheGrind, Callgrind $\rightarrow$ KCacheGrind), Gprof, gperftools, Slurm Workload Manager. 
%\item Web Conferencing Software: WebEx, Vidyo, eZuce, BlueJeans, Zoom.us, WisLine, Skype.  
\item Other: SQL, HTML, Qt, GDB, Cmake.

%\item Other: MySQL, HTML, SQLite, GNU (GPL), QT, GDB, Cmake (CMakeLists.txt), Solidworks, W3schools(HTML, CSS, JAVASCRIPT, 
%SQL, PHP, JQUERY, JSON, PYTHON, XML), OpenSSH, Autoconf, Autotools, Make(file), IPython, Spyder.
\item Some basic experience as sysadmin: Networks on Linux, OpenPBS cluster, Maui Cluster Scheduler.
\end{itemize}
\end{itemize}
%Heel lijstje is wel interessant voor bedrijven:

%- SQL / SQLite
%- QT
%- gdb
%- make / autoconf / autotools
%- Python / IPython
%- json
%- php

%enkel overhouden wat je goed kan. 

%Iedereen heeft bv wel al eens een stukje CSS geschreven, maar wie heeft er echt al mee gespeeld op dagelijkse basis :)

%Python verwacht ik wel een stevige basis van jou (onze discussies in de auto vroeger indachtig)


\section{Work experience}
Lab experience: 
\begin{itemize}
% I contributed to the testing of RPC detectors for the deployment on the endcap wheels of CMS.
\item I contributed to the testing of RPC detectors in the ISR at CERN for the deployment on the endcap wheels of CMS: July - September 2007 (supervisor: Sergey Akimenko (IHEP, Protvino, Russia)) %(07-07-2007 tot 16-08-2007 en 29-08-2007 tot 30-09-2007 , 
\item Construction of IceTop and IceCube detector at South Pole as responsible person for the IceTop detector: January 2010.
\item Construction, build and operate the PMT Test Facility at Meson hall, TRIUMF, Canada at 2 Super-Kamiokande PMTs and the future Hyper-Kamiokande 20'' inch PMT.
\item Construction of mPMT Prototype for E61 and Hyper-Kamiokande at Meson Hall, TRIUMF, Canada.
\end{itemize}
\begin{itemize}
\item Teaching experience - Bachelor students.
\item Supervisor of Master/PhD theses.
\end{itemize}
Courses and training:
\begin{itemize}
%meesterklassen deeltjesfysica te UIA in november 2002 op een zaterdag, (12- 19 or 26 Nov)
\item Masterclasses particle physics for high school students and teachers: Sat Nov. 9, 2002.
\item DESY Summer Student for the HERMES experiment: July - Sept. 2006. %(supervisors: Markus Diefenthaler and Dietmar Zeiler): I worked on a project related to first tracking with the silicon subdetector of the recently installed Recoil detector at the HERMES detector.
\item CERN Summer Student for the CMS experiment: July - Sept. 2007. 
\item Joint Belgian Dutch German Summerschool, Texel, The Netherlands, Sept. 2009: Lectures aimed for graduate students in experimental high energy physics.
\item IceCube software (IceTray) bootcamp, Berlin, Germany, August 2008.
\item 1st IDPASC School, Sesimbra, Portugal, Dec. 2010: Lectures on astroparticle physics, cosmology, statistics and quantum field theory.
\item CORSIKA School, Freudenstadt-Lauterbad, Germany, Dec. 2008: The CORSIKA package is the MC generator for simulating cosmic ray air showers.
\item IceCube software (IceTray) bootcamp, Brussels, Belgium, Sept 2012 as speaker.
\item Several stays abroad (Madison (WI), USA and at the Bartol Research Institute, Newark (DE), USA) for periods of 1-2 months to collaborate with local experts for a total of 6 months.
\item 2014 Neutrino Generator School at the University of Liverpool, UK, 14-16 May 2014. %: The Neutrino Generator school will provide a series of lectures covering a broad range of neutrino interaction phenomenology topics focussing on the connections between theory, experiment and MC simulations. The school will also offer extensive hands-on tutorials of the GENIE and NuWro MC generators.
\item Talk at TRIUMF, Vancouver: Neutrino Oscillations at Nov. 23, 2016 as speaker (Co-op students).
\item WestGrid Research Computing Summer School at UBC, Vancouver, Canada, 11-14 June 2018. %Courses will explore introductory and advanced topics in high performance and cloud computing, parallel programming, databases, bioinformatics, CUDA, Matlab, Python, R and scientific visualization. 
\item TRIUMF Training
\item UBC Training
\end{itemize}

\section{Social experience}
\begin{itemize}
%\item Language course camp French for 12 days at Marche-en-Famenne, July 1998 (with Roeland). 
\item Practice and teach judo: 25 years experience, 10 years teaching 
%\item 2002 - 2008: Leader at Summer camps for 11-18 year old kids.% (with the christian social security, CM).
\end{itemize}
%VVN(1 jaar als actief bestuurslid, 
%1 jaar als vice-voorzitter (waarbij voorzitter een half semester in het buitenland zat), 
%WYP( meegewerkt aan Spektakelshow), 
%CM (monitor sinds 2002, +bergmonicursus+ in juli 2005), 
%taalcursus te Marche-en-Famenne voor 12 dagen bij Roeland(na 1ste middelbaar), 
%BMC+, 
%Judo (2de dan, in het jaar 2006-2007: 14de jaar + initiator), 
%reis naar Desy (met VVN), 
%reis naar CERN (met VUB na het 6de middelbaar, 
%meesterklassen deeltjesfysica te UIA in november 2002 op een zaterdag, 
%zomerschool DESY (26 -07-2006)
%begeleiders : Markus Diefenthaler en Dietmar Zeiler(Universiteit Erlangen beiden), 
%Para junior(juli 2002), 
%Commando junior (juli 2003), 
%lid van ZAP beoordelingscommissie UGent (april/mei 2007) (mag dit in CV?), % Nah
%begeleider/contactpersoon : Sergey Akimenko (Protvino unief, Rusland) en Walter Van Doninck (VUB)

\ignore{
\section{Publications}
*: Significantly contributed.
\subsection{2009-2018}
\begin{enumerate}
\item *\textit{Reconstruction of IceCube coincident events and study of composition-sensitive observables using both the surface and deep detector}, T. Feusels, J. Eisch, C. Xu, for the IceCube Collaboration, \textbf{arXiv:0912.4668v1 [astro-ph.HE], Proceedings of the 31st ICRC, Lodz, Poland, July 2009}
\item *\textit{The IceCube Neutrino Observatory III: Cosmic Rays}, IceCube Collaboration: R. Abbasi et al, \textbf{arXiv:1111.2735 [astro-ph], Proceedings of the 32nd ICRC, Beijing, China, Aug 2011}
\item *\textit{All-particle cosmic ray energy spectrum measured with 26 IceTop stations}, IceCube Collaboration: R. Abbasi et al, \textbf{arXiv:1202.3039 [astro-ph.HE], Astroparticle Physics 44 (2013) 40-58, April 2013}
\item *\textit{Cosmic Ray Composition and Energy Spectrum from 1-30 PeV Using the 40-String Configuration of IceTop and IceCube}, IceCube Collaboration: R. Abbasi et al, \textbf{arXiv: 1207.3455 [astro-ph.HE], Astroparticle Physics 42 (2013) 15-32, Feb 2013}
\item *\textit{IceTop: The surface component of IceCube}, IceCube Collaboration: R. Abbasi et al, \textbf{arXiv:1207.6326 [astro-ph.IM], Nuclear Instruments and Methods A700 (2013) 188-220, 1 February 2013}
\item *\textit{Measurement of cosmic ray energy spectrum with IceTop-73}, IceCube Collaboration: M.G. Aartsen et al, \textbf{arXiv:1307.3795 [astro-ph.HE], Phys. Rev. D 88 (2013) 042004}
\item *\textit{The IceCube Neutrino Observatory Part III: Cosmic Rays}, IceCube Collaboration: M.G. Aartsen et al, \textbf{arXiv:1309.7006 [astro-ph.HE], Proceedings of the 33nd ICRC, Rio de Janeiro, Brazil, 2-9 July 2013}
\item \textit{Letter of Intent to Construct a nuPRISM Detector in the J-PARC Neutrino Beamline}, S. Bhadra et al, \textbf{arXiv:1412.3086 [physics.ins-det]}
\item *\textit{Latest Results on Cosmic Ray Spectrum and Composition from Three Years of IceTop and IceCube}, K. Rawlins, T. Feusels for the IceCube Collaboration, \textbf{arXiv:1510.05225v2 [astro-ph.HE] 9 Nov 2015, Proceedings of the 34th ICRC, The Hague, The Netherlands, July-Aug 2015, PoS (ICRC2015) 0334}
\item *\textit{Measurement of double-differential muon neutrino charged-current interactions on C$_8$H$_8$ without pions in the final state using the T2K off-axis beam}, T2K Collaboration: K. Abe et al, \textbf{Phys. Rev. D 93, 112012 (2016)}
\item *\textit{Hyper-Kamiokande Design Report}, Hyper-Kamiokande Proto-Collaboration: K. Abe et al, \textbf{KEK Preprint 2016-21, ICRR-Report-701-2016-1, available at \url{https://lib-extopc.kek.jp/preprints/PDF/2016/1627/1627021.pdf}, arXiv: 1805.04163 [physics.ins-det]}
\item *\textit{Current status of final-state interaction models and their impact on neutrino-nucleus interactions}, W.Y. Ma, E.S. Pinzon Guerra, M. Yu, A. Fiorentini, T. Feusels for the T2K Collaboration, \textbf{J.Phys.Conf.Ser. 888 (2017) no.1, 012171, \href{https://doi.org/10.1088/1742-6596/888/1/012171}{10.1088/1742-6596/888/1/01217}}
\item *\textit{First measurement of the $\nu_\mu$ charged-current cross section without pions in the final state on a water target}, T2K Collaboration: K. Abe et al, \textbf{Phys. Rev. D 97, 012001 (2018), \href{https://doi.org/10.1103/PhysRevD.97.012001}{10.1103/PhysRevD.97.012001}}
\item *\textit{Characterisation of nuclear effects in muon-neutrino scattering on hydrocarbon with a measurement of final-state kinematics and correlations in charged-current pionless interactions at T2K}, T2K Collaboration: K. Abe et al, \textbf{Phys. Rev. D 98, 032003 (2018), \href{https://doi.org/10.1103/PhysRevD.98.032003}{10.1103/PhysRevD.98.032003}} %\textbf{arXiv:1802.05078 [hep-ex]}
\item *\textit{Using world $\pi^{\pm}$--nucleus scattering data to constrain an intranuclear cascade model}, ES Guerra, C Wilkinson et al, \textbf{arXiv:1812.06912 [hep-ex]}, Submitted in Phys. ReV. D.
%\item \textit{}, T2K Collaboration: K. Abe et al, \textbf{}
%\item \textit{}, Hyper-Kamiokande Proto-Collaboration: K. Abe et al, \textbf{}
\item *\textit{Cosmic Ray Spectrum and Composition from PeV to EeV Using 3 Years of Data From IceTop and IceCube} 
\end{enumerate}




\section{Conferences, Workshops and Seminars}
\begin{itemize}
\item International Cosmic Ray Conference (ICRC) at \L\'{o}d\'{z}, Poland, July 2009: talk contribution.
\item Joint Belgian Dutch German Summerschool, Texel, The Netherlands, Sept. 2009: talk contribution.
\item Cosmic Ray Workshop from 22 February 2010 to 24 February 2010 at DESY, Zeuthen: talk contribution.
\item European Cosmic Ray Symposium (ECRS) at Turku, Finland, August 2010: poster contribution.
\item 1st IDPASC School, Sesimbra, Portugal, Dec. 2010: talk contribution.
\item European IceTop/RASTA Workshop from 10 February 2011 to 11 February 2011 at Gent: organization and several talks.
\item Cosmic Ray Workshop at the Bartol Research Institute, Newark (DE), USA, June 2011
\item International Cosmic Ray Conference (ICRC) at Beijing, China, Aug 2011: poster contribution.
\item 3rd Workshop for Air Shower Detection at High Altitudes at the IPN-Orsay 6-7 October 2011: invited talk.
\item Cosmic Ray Workshop at the Bartol Research Institute, Newark (DE), USA, June 2012: contributed several talks.
\item IceCube software (IceTray) bootcamp, Brussels, Sept. 2012: speaker. % (shared the day with Emanuel)
\item International Cosmic Ray Conference (ICRC) at Rio De Janeiro, Brazil, July 2013: talk and poster contribution. 
%% my plots also shown at VLVnT 2013, TAUP 2013, TeVPA 2013
% collaboration meetings?? Plenary!! 2

\item 9th International Workshop on Neutrino-Nucleus Interactions in the Few-GeV Region (NuInt14) at the Surrey, UK, 19-24 May 2014. 
\item TRIUMF Particle Phys/SciTec Seminar, May 8, 2015: talk seminar (Photosensor Test Facility).
\item CC0pi and MEC Summer 2015 Workshop, July 8-10 2015, TRIUMF, Vancouver, Canada: Kendall Mahn, Jiae Kim, Akira Konaka, Hirohisa Tanaka and I organize it.
\item NuFact15: XVII International Workshop on Neutrino Factories and Future Neutrino Facilities at the Rio De Janeiro, Brazil, 10-15 August 2015: 2 talks contribution (one for T2K, one for Hyper-Kamiokande)
\item 2016 CAP Congress at the University of Ottawa, SITE Building (Ottawa, Ontario): 13-17 June 2016: talk contribution (T2K-II)
\item Neutrino 2016 at London, UK, July 4-9 2016: poster contribution (mPMT)
\item mPMT/NEUT workshop at Nikhef, Amsterdam, Netherlands, 14-16 July 2016: organizer and several talks.%and Dorothea
\item T2K Cross Sections Pre-Collaboration Meeting, Sept 22-24 2016, Tokai, Japan: Talk contribution.
\item International Workshop on Next Generation Nucleon Decay and Neutrino Detectors 2016 at IHEP, Beijing, China, Nov. 3-5, 2016: talk contribution (Hyper-Kamiokande) 
\item NSERC Review, Toronto, Canada, Dec. 13, 2016: talk contribution (Photosensor R\&D for Super-K, T2K, E61 (NuPRISM) and Hyper-Kamiokande) 
\item E61(NuPRISM) Workshop, Toronto, Canada, Dec. 14, 2016: talk contribution (Photosensor R\&D) 
\item SNOLab Seminar, Dec. 15, 2016: talk seminar (T2K, E61 (NuPRISM) and Hyper-Kamiokande)
\item UGent Seminar, Jan. 9, 2017: talk seminar (T2K, E61 (NuPRISM) and Hyper-Kamiokande)

\end{itemize}
}
%lots of collaboration Meetings: PLENARY 
\ignore{
\section{Outreach activities}
\begin{itemize}
\item Organization of VVN lecture for a broad audience about South Pole and IceCube physics, with live weblink with two colleagues at South Pole (Dec. 7, 2009).
\item Interview about IceCube and life at South Pole in Schamper magazine, the UGent students magazine.
\item Interview about IceCube and life at South Pole in EOS, January 2009.
\item Visiting my high school and gave a talk about IceCube and life at South Pole, November 2009.
\item Gave an invited talk for amateur astronomers (Volkssterrenwacht Urania) about Neutrino Astronomy, IceCube and life at South Pole with a live weblink with a colleague at South Pole (Nov. 2010)
\item Started a blog in 2009-2010 with 2 UGent colleagues when the three of us were all going to work at South Pole on the construction of the IceCube experiment: http://www.naardezuidpool.ugent.be/home.html.
\item Gave an invited talk for amateur astronomers (Volkssterrenwacht Urania) in Nov 2012. (Part 2 of the talk from Nov. 2010)
\end{itemize}
}
\ignore{
Outreach : VVN activity
           Schamper
           Ugent Magazine
	   EOS
	   Pers
	   Schoolbezoek
	   Volkssterrenwacht Urania (Nov 2010) met Arne
           IceCube BLOG
}
\ignore{Motivation for DESY: Since I was young, I already had a great interest in science and research. 
I always said to my parents that I wanted to work in a laboratory later and discover something new. 
In elementary school I found out that my greatest interest went out to particles. I collected minerals
 and I wanted to know what they were made off, what molecules were and things like that. 
When I was sixteen, seventeen years old I realized that the real study of molecules, atoms and 
particles and what the real building blocks of matter were, is not really chemistry, but physics. 
I also started to get interested in quantum mechanics, relativity, elementary particle physics and 
the Standard Model. I quickly decided that it had to go study physics because I wanted to know all
 those theories in detail. In my last year of high school students could follow 
(an introductory course of one day on the Standard Model at the University of Antwerp). It was 
really interesting and there we had to analyse a few electron-positron collisions to decide what 
products were formed. In the summer of 2003 I went on a trip to CERN with the University of 
Brussels (VUB). This was a trip for students of 17 and 18 years old. It was very fascinating and I
 hoped that I could work in such a laboratory later. All these activities contributed even more to 
my interest in physics and my desire to really work in such a laboratory.
It have
theoretical, for my thesis or perhaps for a Ph D. That
 Then I can really work in a research group and experience what it really is to study and work for 
2 months on an experiment or a problem. I would give up everything in my summer vacation to be able 
to do something that exciting. Also the contact with lots of foreign students is a reason for me 
to apply. I think and hope it will be an enriching experience.}

\end{document}
